\documentclass[aspectratio=169]{beamer}
\usepackage[all]{xy}
\usetheme{metropolis}

\usepackage{graphicx}
\usepackage{bm}
\usepackage{amsmath}
\usepackage{beamerfoils}
\usepackage{hyperref}
\usepackage{listings}
\usepackage{textpos}
\usepackage{alltt}
\usepackage{color}
\usepackage{tikz}
\usepackage{pgfplots}
\pgfplotsset{compat=newest,
   /pgfplots/ybar legend/.style={
    /pgfplots/legend image code/.code={%
       \draw[##1,/tikz/.cd,yshift=-0.25em]
        (0cm,0cm) rectangle (2pt,0.8em);},
   },
}

\definecolor{color1}{rgb}{0.549,0.815,0.960}
\definecolor{color2}{rgb}{0.607,0.937,0.498}
\definecolor{color3}{rgb}{0.839,0.929,0.427}
\definecolor{color4}{rgb}{0.878,0.619,0.141}
\definecolor{color5}{rgb}{0.396,0.165,0.055}
\definecolor{color6}{rgb}{0.643,0.368,0.898}
\definecolor{colortop}{rgb}{0.878,0.141,0.141}

\usepackage{wasysym}

\newcommand{\snest}{Nest}

\newcommand{\extrabold}{}%{\bfseries}

  \lstdefinelanguage{diff}{
	basicstyle=\ttfamily\extrabold\scriptsize,
	morecomment=[f][\color{subtitlex}]{@},
	morecomment=[f][\color{gr}]{+},
	morecomment=[f][\color{red}]{-},
	morecomment=[f][\color{purple}]{*},
	morecomment=[f][\color{purple}]{?},
        keepspaces=true,
	escapechar=£,
	identifierstyle=\color{black},
  }

  \lstdefinelanguage{tdiff}{
	basicstyle=\ttfamily\extrabold\tiny,
	morecomment=[f][\color{subtitlex}]{@},
	morecomment=[f][\color{gr}]{+},
	morecomment=[f][\color{red}]{-},
        keepspaces=true,
	escapechar=£,
	identifierstyle=\color{black},
  }

\lstloadlanguages{C}
\lstset{language=C,
	basicstyle=\ttfamily\extrabold\scriptsize,
	backgroundcolor=\color{white},
        showspaces=false,
        rulesepcolor=\color{gray},
	showstringspaces=false,
	keywordstyle=\bfseries\color{blue!40!black},
	commentstyle=\itshape\color{purple!40!black},
	identifierstyle=\color{black},
	stringstyle=\color{red},
        numbers=none,
        numbersep=2pt,
        morekeywords={elif},
       }

%\lstset{escapeinside={(*@}{@*)},style=customP}

\newcommand{\ttlb}{\mbox{\tt \char'173}}
\newcommand{\ttrb}{\mbox{\tt \char'175}}
\newcommand{\ttbs}{\mbox{\tt \char'134}}
\newcommand{\ttmid}{\mbox{\tt \char'174}}
\newcommand{\tttld}{\mbox{\tt \char'176}}
\newcommand{\ttcar}{\mbox{\tt \char'136}}
\newcommand{\msf}[1]{\mbox{\sf{{#1}}}}
\newcommand{\mita}[1]{\mbox{\it{{#1}}}}
\newcommand{\mbo}[1]{\mbox{\bf{{#1}}}}
\newcommand{\mth}[1]{\({#1}\)}
\newcommand{\ssf}[1]{\mbox{\scriptsize\sf{{#1}}}}
\newcommand{\sita}[1]{\mbox{\scriptsize\it{{#1}}}}
\newcommand{\mrm}[1]{\mbox{\rm{{#1}}}}
\newcommand{\mtt}[1]{\mbox{\tt{{#1}}}}

\definecolor{gr}{rgb}{0.22,0.63,0.08}
\definecolor{subtitlex}{rgb}{0.2,0.2,0.7}

\setlength{\fboxrule}{1.75pt}
\newcommand{\mybox}[2]{\fcolorbox{#1}{black!2}{\makebox[1.2\height]{\begin{tabular}{c}\textcolor{#1}{#2}\end{tabular}}}}
\newcommand{\emptybox}[1]{\mybox{#1}{\textcolor{black!2}{t1}}}
\newcommand{\blackbox}{\emptybox{black}}
\newcommand{\grbox}[1]{\mybox{gr}{T{#1}}}
\newcommand{\redbox}[1]{\mybox{subtitlex}{\textcolor{red}{T{#1}}}}
\newcommand{\gredbox}[1]{\mybox{gr}{\textcolor{red}{T{#1}}}}
\newcommand{\bluebox}[1]{\mybox{subtitlex}{T{#1}}}

%\\[-2mm]
%5\\[-2mm]{}

\title{Part of cocci\_grep, as used in get\_constants}
\author{Julia Lawall, Inria}
\date{June 5, 2023\\ \mbox{}
}

\begin{document}

\frame{\titlepage}

\frame[containsverbatim]{\frametitle{Get\_constants}

\textcolor{subtitlex}{Goal:} Collect the tokens in a semantic patch that must be
  present for the semantic patch to match.

\begin{lstlisting}[language=tdiff]
@@
type t;
expression x,e;
@@
(
  x = kzalloc(sizeof(t), ...);
|
  x = kcalloc(sizeof(t), ...);
)
... when != x = e
- memset(x, 0, sizeof(t));
\end{lstlisting}

{\tt Get\_constants} produces $({\tt kzalloc} \vee {\tt kcalloc}) \wedge {\tt memset}$.
\begin{itemize}
\item The result is in conjunctive normal form (CNF).
\end{itemize}}

\frame[containsverbatim]{\frametitle{Basic data types}

\textcolor{subtitlex}{Observations:}
\begin{itemize}
\setlength{\itemsep}{3mm}
\item We only collect strings matched by the semantic patch (no negation).
\item A list of lists of strings is sufficient:
\begin{lstlisting}[language=C]
[[kzalloc; kcalloc]; [memset]]
\end{lstlisting}
\item In Rust:
\begin{lstlisting}[language=C]
type Clause<'a> = BTreeSet<&'a str>;
type CNF<'a> = BTreeSet<Clause<'a>>;
\end{lstlisting}
\end{itemize}

\vfill

\textcolor{subtitlex}{Goal:} Simplify the formulas, for
better performance with indexing tools.
}

\frame[containsverbatim]{\frametitle{Problem 1: Count occurrences of each string
    in a formula}

\textcolor{subtitlex}{Signature:}
\begin{lstlisting}[language=diff]
fn count_atoms<'a>(l: &CNF<'a>) -> Vec<(&'a str,u32)> { }
\end{lstlisting}

\vspace{\baselineskip}

\textcolor{subtitlex}{Steps:}
\begin{enumerate}
\setlength{\itemsep}{3mm}
\item Compute counts in a hash table.
\item Convert the hash table to a vector with the final results.
\item Sort the vector by increasing counts.
\end{enumerate}
}

\frame[containsverbatim]{\frametitle{Compute counts in a hash table}

\begin{lstlisting}[language=diff]
fn count_atoms<'a>(l: &CNF<'a>) -> Vec<(&'a str,u32)> { }
    let mut tbl = HashMap::new();
    for x in l.iter().flatten() {
        match tbl.get_mut(x) {
            None => { tbl.insert(x,1); }
            Some(old) => { *old = *old + 1; }
        }
    };
    ...
}
\end{lstlisting}

\vfill
\begin{itemize}
\item Should there be \verb+&+ be on any uses of {\tt x}?
\item Is {\tt iter()} and then {\tt flatten()} the right choice?
\end{itemize}
}

\frame[containsverbatim]{\frametitle{Convert the hash table to a vector
    with the final results}

\begin{lstlisting}[language=diff]
fn count_atoms<'a>(l: &CNF<'a>) -> Vec<(&'a str,u32)> { }
    let mut tbl = HashMap::new();
    ... // see previous slide
    let mut res : Vec<(&'a str,u32)> = tbl.into_iter().map(|(&s,ct)| (s,ct)).collect();
    ...
}
\end{lstlisting}

\vfill
\begin{itemize}
\item Why is {\tt into\_iter} needed ({\tt iter} gives a type error)?
\item Is the {\tt map} necessary?
\item Could this be done by folding over the hash table and adding each key
  and value to the {\tt res} vector?
\begin{itemize}
\item[--] HashMaps have methods for accessing the keys and the values, but
  not for iterating over both.
\end{itemize}
\end{itemize}
}

\frame[containsverbatim]{\frametitle{Sorting the results by the count
    (complete implementation)}

\begin{lstlisting}[language=diff]
fn count_atoms<'a>(l: &CNF<'a>) -> Vec<(&'a str,u32)> {
    let mut tbl = HashMap::new();
    for x in l.iter().flatten() {
        match tbl.get_mut(x) {
            None => { tbl.insert(x,1); }
            Some(old) => { *old = *old + 1; }
        }
    };
    £\textcolor{red}{\tt{let mut res : Vec<(\&'a str,u32)> = tbl.into\_iter().map(|(\&s,ct)| (s,ct)).collect();}}£
    £\textcolor{red}{\tt{res.sort\_by(|(\_,ct1), (\_,ct2)| ct1.cmp(ct2));}}£
    £\textcolor{red}{\tt{res}}£
}
\end{lstlisting}

\vfill
\begin{itemize}
\item Could the last three lines be more concise?
\end{itemize}
}

\frame[containsverbatim]{\frametitle{Problem 2: Extend a set of clauses (CNF) in
    some way with an element}

\textcolor{subtitlex}{Signature:}
\begin{lstlisting}[language=diff]
fn extend<'a>(element : &'a str, res : &Clause<'a>, available : &mut CNF<'a>)
    -> Clause<'a> { }
\end{lstlisting}

\textcolor{subtitlex}{Steps:}
\begin{enumerate}
\setlength{\itemsep}{3mm}
\item Partition {\tt available} into the clauses that contain the element
  and those that do not.
\item Flatten the set of clauses that contain the element into a single
  clause, {\tt added}.
\item Remove the clauses of {\tt available} that are a subset of {\tt
  added}.
\item Union the {\tt added} clause with {\tt res}, as the result.
\end{enumerate}}

\frame[containsverbatim]{\frametitle{Partitioning and flattening, to produce {\tt added}}

\textcolor{subtitlex}{OCaml:}
\begin{lstlisting}[language=diff]
let (added,available) = List.partition (List.mem element) available in
let added = List.fold_left (fun res elem -> Common.union_set elem res) [] added in
\end{lstlisting}

\vfill

\textcolor{subtitlex}{Rust:}
\begin{lstlisting}[language=diff]
fn extend<'a>(element : &'a str, res : &Clause<'a>, available : &mut CNF<'a>)
    -> Clause<'a> {
    let mut added : Clause<'a> = BTreeSet::new();
    available
        .retain(|l|
                !(l.contains(element)) || { l.iter().for_each(|x| { added.insert(x); }); false });
    ...
}
\end{lstlisting}

\vfill
\begin{itemize}
\item Surely there is a better approach to partition {\tt l}?
\end{itemize}}

\frame[containsverbatim]{\frametitle{Remove clauses made redundant by {\tt
      added}}

\begin{lstlisting}[language=diff]
fn extend<'a>(element : &'a str, res : &Clause<'a>, available : &mut CNF<'a>)
    -> Clause<'a> {
    let mut added : Clause<'a> = BTreeSet::new();
    available
        .retain(|l| !(l.contains(element)) || { l.iter().for_each(|x| { added.insert(x); }); false });
    £\textcolor{red}{\tt{available.retain(|l| !(l.is\_subset(\&added)));}}£
    ...
}
\end{lstlisting}
}

\frame[containsverbatim]{\frametitle{Combine {\tt added} and {\tt res}
    (complete implementation)}

\begin{lstlisting}[language=diff]
fn extend<'a>(element : &'a str, res : &Clause<'a>, available : &mut CNF<'a>) -> Clause<'a> {
    let mut added : Clause<'a> = BTreeSet::new();
    available
        .retain(|l| !(l.contains(element)) || { l.iter().for_each(|x| { added.insert(x); }); false });
    available.retain(|l| !(l.is_subset(&added)));
    £\textcolor{red}{\tt{added.union(\&res).cloned().collect()}}£
}
\end{lstlisting}

\vfill

\begin{itemize}
\item Why {\tt cloned()}?
\end{itemize}}

\frame[containsverbatim]{\frametitle{Problem 3: Split the clauses according
  the presence of the least occurring and most occurring strings}

\textcolor{subtitlex}{Signature:}
\begin{lstlisting}[language=diff]
fn leftres_rightres<'a>(front : Vec<((&'a str,u32),usize)>,
                        back : Vec<((&'a str,u32),usize)>,
                        available : &mut CNF<'a>)
    -> (Clause<'a>,Clause<'a>) { }
\end{lstlisting}

\textcolor{subtitlex}{Steps:}
\begin{enumerate}
\setlength{\itemsep}{3mm}
\item Create two accumulators, {\tt leftres} and {\tt rightres}.
\item Iterate over the list of strings and its reverse until reaching the
  middle.
\item Use {\tt extend} to build up {\tt leftres} and {\tt rightres} for
  each string.
\end{enumerate}}

\frame[containsverbatim]{\frametitle{Create the accumulators}

\begin{lstlisting}[language=diff]
fn leftres_rightres<'a>(front : Vec<((&'a str,u32),usize)>,
                        back : Vec<((&'a str,u32),usize)>,
                        available : &mut CNF<'a>)
    -> (Clause<'a>,Clause<'a>) {
    £\textcolor{red}{\tt{let mut leftres~~: Clause<'a> = BTreeSet::new();}}£
    £\textcolor{red}{\tt{let mut rightres : Clause<'a> = BTreeSet::new();}}£
    ...
    (leftres,rightres)
}
\end{lstlisting}}

\frame[containsverbatim]{\frametitle{Set up the iteration}

\begin{lstlisting}[language=diff]
fn leftres_rightres<'a>(front : Vec<((&'a str,u32),usize)>,
                        back : Vec<((&'a str,u32),usize)>,
                        available : &mut CNF<'a>)
    -> (Clause<'a>,Clause<'a>) {
    let mut leftres  : Clause<'a> = BTreeSet::new();
    let mut rightres : Clause<'a> = BTreeSet::new();
    £\textcolor{red}{\tt{front.iter().enumerate().}}£
        £\textcolor{red}{\tt{for\_each(|(pos,\&((f,\_),nf))| \ttlb}}£
                 £\textcolor{red}{\tt{let ((b,\_),nb) = back[pos];}}£
                 ... £\textcolor{red}{\tt{\ttrb );}}£
    (leftres,rightres)
}
\end{lstlisting}}

\frame[containsverbatim]{\frametitle{If no more clauses are available, then
  abort}

\begin{lstlisting}[language=diff]
fn leftres_rightres<'a>(front : Vec<((&'a str,u32),usize)>,
                        back : Vec<((&'a str,u32),usize)>,
                        available : &mut CNF<'a>)
    -> (Clause<'a>,Clause<'a>) {
    let mut leftres  : Clause<'a> = BTreeSet::new();
    let mut rightres : Clause<'a> = BTreeSet::new();
    front.iter().enumerate().
        for_each(|(pos,&((f,_),nf))| {
                 let ((b,_),nb) = back[pos];
                 £\textcolor{red}{\tt{if available.is\_empty() \ttlb}}£
                     £\textcolor{red}{\tt{return ();}}£
                 £\textcolor{red}{\tt{\ttrb}}£
                 £\textcolor{red}{\tt{else}}£ ... } );
    (leftres,rightres)
}
\end{lstlisting}}

\frame[containsverbatim]{\frametitle{Using {\tt extend} (complete implementation)}

\begin{lstlisting}[language=tdiff]
fn leftres_rightres<'a>(front : Vec<((&'a str,u32),usize)>,
                        back : Vec<((&'a str,u32),usize)>,
                        available : &mut CNF<'a>)
    -> (Clause<'a>,Clause<'a>) {
    let mut leftres  : Clause<'a> = BTreeSet::new();
    let mut rightres : Clause<'a> = BTreeSet::new();
    front.iter().enumerate().
        for_each(|(pos,&((f,_),nf))| {
                 let ((b,_),nb) = back[pos];
                 if available.is_empty() {
                     return (); // abort the for_each
                 }
                 else {
                     £\textcolor{red}{\tt{if nf < nb \ttlb}}£
                         £\textcolor{red}{\tt{leftres = extend(f, \&leftres, available);}}£
                         £\textcolor{red}{\tt{rightres = extend(b, \&rightres, available);}}£
                     £\textcolor{red}{\tt{\ttrb}}£
                     £\textcolor{red}{\tt{else if nf == nb \ttlb}}£
                         £\textcolor{red}{\tt{available.iter().flatten().for\_each(|\&x| { leftres.insert(x); });}}£
                     £\textcolor{red}{\tt{\ttrb}}£
                 } } );
    (leftres,rightres)
}
\end{lstlisting}

\vfill
\begin{itemize}
\item Mutability: overwriting a variable vs.~overwriting the contents of
  the variable.
\end{itemize}
}

\frame[containsverbatim]{\frametitle{Problem 4: Putting it all together}

\begin{lstlisting}[language=diff]
fn split<'a>(l : &CNF<'a>) -> CNF<'a> {
    let mut tbl = count_atoms(l);
    let mut pretbl : Vec<&'a str> = Vec::new();
    tbl.retain(|&(s,ct)| ct > 1 || { pretbl.insert(0,&s); false });
    let mut available = l.clone();
    let mut preres : CNF<'a> = CNF::new();
    for f in pretbl.iter() {
        let res : Clause<'a> = extend(&f, &BTreeSet::new(), &mut available);
        if !(res.is_empty()) {
            preres.insert(res);
        }
    };
    let ltbl : Vec<_> = tbl.iter().enumerate().map(|(pos,&e)| (e,pos)).collect();
    let mut rtbl = ltbl.clone();
    rtbl.reverse();
    let _ = leftres_rightres(ltbl,rtbl,&mut available);
    if !leftres.is_empty() { preres.insert(leftres); }
    if !rightres.is_empty() { preres.insert(rightres); }
    preres
}
\end{lstlisting}
}

\frame[containsverbatim]{\frametitle{Count strings, and extract
    those that occur only once}

\begin{lstlisting}[language=tdiff]
fn split<'a>(l : &CNF<'a>) -> CNF<'a> {
    £\textcolor{red}{\tt{let mut tbl = count\_atoms(l);}}£
    £\textcolor{red}{\tt{let mut pretbl : Vec<\&'a str> = Vec::new();}}£
    £\textcolor{red}{\tt{tbl.retain(|\&(s,ct)| ct > 1 || \ttlb~pretbl.insert(0,\&s); false~\ttrb);}}£
    let mut available = l.clone();
    let mut preres : CNF<'a> = CNF::new();
    for f in pretbl.iter() {
        let res : Clause<'a> = extend(&f, &BTreeSet::new(), &mut available);
        if !(res.is_empty()) {
            preres.insert(res);
        }
    };
    let ltbl : Vec<_> = tbl.iter().enumerate().map(|(pos,&e)| (e,pos)).collect();
    let mut rtbl = ltbl.clone();
    rtbl.reverse();
    let _ = leftres_rightres(ltbl,rtbl,&mut available);
    if !leftres.is_empty() { preres.insert(leftres); }
    if !rightres.is_empty() { preres.insert(rightres); }
    preres
}
\end{lstlisting}

\begin{itemize}
\item[] \textcolor{black!2}{Why is the {\tt \&} needed in the call to {\tt map} (without it, the
  tuple is a reference)?}
\end{itemize}
}

\frame[containsverbatim]{\frametitle{Process strings that occur only once
    (consumes {\tt available})}

\begin{lstlisting}[language=tdiff]
fn split<'a>(l : &CNF<'a>) -> CNF<'a> {
    let mut tbl = count_atoms(l);
    let mut pretbl : Vec<&'a str> = Vec::new();
    tbl.retain(|&(s,ct)| ct > 1 || { pretbl.insert(0,&s); false });
    £\textcolor{red}{\tt{let mut available = l.clone();}}£
    £\textcolor{red}{\tt{let mut preres : CNF<'a> = CNF::new();}}£
    £\textcolor{red}{\tt{for f in pretbl.iter() \ttlb}}£
        £\textcolor{red}{\tt{let res : BTreeSet<\&'a str> = extend(\&f, \&BTreeSet::new(), \&mut available);}}£
        £\textcolor{red}{\tt{if !(res.is\_empty()) \ttlb}}£
            £\textcolor{red}{\tt{preres.insert(res);}}£
        £\textcolor{red}{\tt{\ttrb}}£
    £\textcolor{red}{\tt{\ttrb;}}£
    let ltbl : Vec<_> = tbl.iter().enumerate().map(|(pos,&e)| (e,pos)).collect();
    let mut rtbl = ltbl.clone();
    rtbl.reverse();
    let _ = leftres_rightres(ltbl,rtbl,&mut available);
    if !leftres.is_empty() { preres.insert(leftres); }
    if !rightres.is_empty() { preres.insert(rightres); }
    preres
}
\end{lstlisting}

\begin{itemize}
\item[] \textcolor{black!2}{Why is the {\tt \&} needed in the call to {\tt map} (without it, the
  tuple is a reference)?}
\end{itemize}
}

\frame[containsverbatim]{\frametitle{Make indices explicit in {\tt tbl},
    call {\tt leftres\_rightres}}

\begin{lstlisting}[language=tdiff]
fn split<'a>(l : &CNF<'a>) -> CNF<'a> {
    let mut tbl = count_atoms(l);
    let mut pretbl : Vec<&'a str> = Vec::new();
    tbl.retain(|&(s,ct)| ct > 1 || { pretbl.insert(0,&s); false });
    let mut available = l.clone();
    let mut preres : CNF<'a> = CNF::new();
    for f in pretbl.iter() {
        let res : BTreeSet<&'a str> = extend(&f, &BTreeSet::new(), &mut available);
        if !(res.is_empty()) {
            preres.insert(res);
        }
    };
    £\textcolor{red}{\tt{let ltbl : Vec<\_> = tbl.iter().enumerate().map(|(pos,\&e)| (e,pos)).collect();}}£
    £\textcolor{red}{\tt{let mut rtbl = ltbl.clone();}}£
    £\textcolor{red}{\tt{rtbl.reverse();}}£
    £\textcolor{red}{\tt{let \_ = leftres\_rightres(ltbl,rtbl,\&mut available);}}£
    if !leftres.is_empty() { preres.insert(leftres); }
    if !rightres.is_empty() { preres.insert(rightres); }
    preres
}
\end{lstlisting}

\begin{itemize}
\item Why is the {\tt \&} needed in the call to {\tt map} (without it, the
  tuple is a reference)?
\end{itemize}
}

\frame[containsverbatim]{\frametitle{Combining the results}

\begin{lstlisting}[language=tdiff]
fn split<'a>(l : &CNF<'a>) -> CNF<'a> {
    let mut tbl = count_atoms(l);
    let mut pretbl : Vec<&'a str> = Vec::new();
    tbl.retain(|&(s,ct)| ct > 1 || { pretbl.insert(0,&s); false });
    let mut available = l.clone();
    let mut preres : CNF<'a> = CNF::new();
    for f in pretbl.iter() {
        let res : BTreeSet<\&'a str> = extend(&f, &BTreeSet::new(), &mut available);
        if !(res.is_empty()) {
            preres.insert(res);
        }
    };
    let ltbl : Vec<_> = tbl.iter().enumerate().map(|(pos,&e)| (e,pos)).collect();
    let mut rtbl = ltbl.clone();
    rtbl.reverse();
    let _ = leftres_rightres(ltbl,rtbl,&mut available);
    £\textcolor{red}{\tt{if !leftres.is\_empty() \ttlb preres.insert(leftres); \ttrb}}£
    £\textcolor{red}{\tt{if !rightres.is\_empty() \ttlb preres.insert(rightres); \ttrb}}£
    £\textcolor{red}{\tt{preres}}£
}
\end{lstlisting}

\hfill

\begin{itemize}
\item Is {\tt preres} getting copied in returning the result?
\end{itemize}
}
\end{document}
