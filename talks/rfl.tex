\documentclass[aspectratio=169]{beamer}
\usepackage[all]{xy}
\usetheme{metropolis}

\usepackage{graphicx}
\usepackage{bm}
\usepackage{amsmath}
\usepackage{beamerfoils}
\usepackage{hyperref}
\usepackage{listings}
\usepackage{textpos}
\usepackage{alltt}
\usepackage{color}
\usepackage{tikz}
\usepackage{pgfplots}
\pgfplotsset{compat=newest,
   /pgfplots/ybar legend/.style={
    /pgfplots/legend image code/.code={%
       \draw[##1,/tikz/.cd,yshift=-0.25em]
        (0cm,0cm) rectangle (2pt,0.8em);},
   },
}

\definecolor{color1}{rgb}{0.549,0.815,0.960}
\definecolor{color2}{rgb}{0.607,0.937,0.498}
\definecolor{color3}{rgb}{0.839,0.929,0.427}
\definecolor{color4}{rgb}{0.878,0.619,0.141}
\definecolor{color5}{rgb}{0.396,0.165,0.055}
\definecolor{color6}{rgb}{0.643,0.368,0.898}
\definecolor{colortop}{rgb}{0.878,0.141,0.141}

\usepackage{wasysym}

\newcommand{\snest}{Nest}

\newcommand{\extrabold}{}%{\bfseries}

  \lstdefinelanguage{diff}{
	basicstyle=\ttfamily\extrabold\scriptsize,
	morecomment=[f][\color{subtitlex}]{@},
	morecomment=[f][\color{gr}]{+},
	morecomment=[f][\color{red}]{-},
	morecomment=[f][\color{purple}]{*},
	morecomment=[f][\color{purple}]{?},
        keepspaces=true,
	escapechar=£,
	identifierstyle=\color{black},
  }

  \lstdefinelanguage{tdiff}{
	basicstyle=\ttfamily\extrabold\tiny,
	morecomment=[f][\color{subtitlex}]{@},
	morecomment=[f][\color{gr}]{+},
	morecomment=[f][\color{red}]{-},
        keepspaces=true,
	escapechar=£,
	identifierstyle=\color{black},
  }

\lstloadlanguages{C}
\lstset{language=C,
	basicstyle=\ttfamily\extrabold\scriptsize,
	backgroundcolor=\color{white},
        showspaces=false,
        rulesepcolor=\color{gray},
	showstringspaces=false,
	keywordstyle=\bfseries\color{blue!40!black},
	commentstyle=\itshape\color{purple!40!black},
	identifierstyle=\color{black},
	stringstyle=\color{red},
        numbers=none,
        numbersep=2pt,
        morekeywords={elif},
       }

%\lstset{escapeinside={(*@}{@*)},style=customP}

\newcommand{\ttlb}{\mbox{\tt \char'173}}
\newcommand{\ttrb}{\mbox{\tt \char'175}}
\newcommand{\ttbs}{\mbox{\tt \char'134}}
\newcommand{\ttmid}{\mbox{\tt \char'174}}
\newcommand{\tttld}{\mbox{\tt \char'176}}
\newcommand{\ttcar}{\mbox{\tt \char'136}}
\newcommand{\msf}[1]{\mbox{\sf{{#1}}}}
\newcommand{\mita}[1]{\mbox{\it{{#1}}}}
\newcommand{\mbo}[1]{\mbox{\bf{{#1}}}}
\newcommand{\mth}[1]{\({#1}\)}
\newcommand{\ssf}[1]{\mbox{\scriptsize\sf{{#1}}}}
\newcommand{\sita}[1]{\mbox{\scriptsize\it{{#1}}}}
\newcommand{\mrm}[1]{\mbox{\rm{{#1}}}}
\newcommand{\mtt}[1]{\mbox{\tt{{#1}}}}

\definecolor{gr}{rgb}{0.22,0.63,0.08}
\definecolor{subtitlex}{rgb}{0.2,0.2,0.7}

\setlength{\fboxrule}{1.75pt}
\newcommand{\mybox}[2]{\fcolorbox{#1}{black!2}{\makebox[1.2\height]{\begin{tabular}{c}\textcolor{#1}{#2}\end{tabular}}}}
\newcommand{\emptybox}[1]{\mybox{#1}{\textcolor{black!2}{t1}}}
\newcommand{\blackbox}{\emptybox{black}}
\newcommand{\grbox}[1]{\mybox{gr}{T{#1}}}
\newcommand{\redbox}[1]{\mybox{subtitlex}{\textcolor{red}{T{#1}}}}
\newcommand{\gredbox}[1]{\mybox{gr}{\textcolor{red}{T{#1}}}}
\newcommand{\bluebox}[1]{\mybox{subtitlex}{T{#1}}}

%\\[-2mm]
%5\\[-2mm]{}

\title{Coccinelle for Rust\\
https://gitlab.inria.fr/coccinelle/coccinelleforrust.git
}
\author{Julia Lawall, Tathagata Roy}
\date{July 23, 2023\\ \mbox{}
}

\begin{document}

\frame{\titlepage}
\frame{\frametitle{Goals}

\begin{itemize}
\setlength{\itemsep}{4mm}
\item Perform repetitive transformations at a large scale. \pause
\begin{itemize}
\setlength{\itemsep}{2mm}
\item[--] Rust is 1.6 MLOC.
\item[--] The Linux kernel is 23 MLOC. \pause
\item[--] Collateral evolutions: a change in an API requires changes in all clients.
\end{itemize}

\pause

\item Provide a transformation language that builds on developer expertise.

\pause

\item Changes + developer familiarity = (semantic) patches
\end{itemize}
}

\frame[containsverbatim]{\frametitle{An example change (Rust
    repository)}

\begin{lstlisting}[language=tdiff]
commit d822b97a27e50f5a091d2918f6ff0ffd2d2827f5
Author: Kyle Matsuda <kyle.yoshio.matsuda@gmail.com>
Date:   Mon Feb 6 17:48:12 2023 -0700

    change usages of type_of to bound_type_of

diff --git a/compiler/rustc_borrowck/src/diagnostics/conflict_errors.rs b/compiler/.../conflict_errors.rs
@@ -2592,4 +2592,4 @@ fn annotate_argument_and_return_for_borrow(
             } else {
-                £\textcolor{red}{let ty = }£self.infcx.tcx.type_of(self.mir_def_id())£\textcolor{red}{;}£
+                £\textcolor{gr}{let ty = }£self.infcx.tcx.bound_type_of(self.mir_def_id()).subst_identity()£\textcolor{gr}{;}£
                 match ty.kind() {
                     ty::FnDef(_, _) | ty::FnPtr(_) => self.annotate_fn_sig(
diff --git a/compiler/rustc_borrowck/src/diagnostics/mod.rs b/compiler/.../mod.rs
@@ -1185,4 +1185,4 @@ fn explain_captures(
                         matches!(tcx.def_kind(parent_did), rustc_hir::def::DefKind::Impl { .. })
                             .then_some(parent_did)
-                            £\textcolor{red}{.and\_then(|did| match}£ tcx.type_of(did)£\textcolor{red}{.kind() \ttlb}£
+                            £\textcolor{gr}{.and\_then(|did| match}£ tcx.bound_type_of(did).subst_identity()£\textcolor{gr}{.kind() \ttlb}£
                                 ty::Adt(def, ..) => Some(def.did()),
...
\end{lstlisting}

\textcolor{subtitlex}{136 files changed, 385 insertions(+), 262 deletions(-)}
}

\frame[containsverbatim]{\frametitle{An example change (Rust
    repository)}

\begin{lstlisting}[language=tdiff]
commit d822b97a27e50f5a091d2918f6ff0ffd2d2827f5
Author: Kyle Matsuda <kyle.yoshio.matsuda@gmail.com>
Date:   Mon Feb 6 17:48:12 2023 -0700

    change usages of type_of to bound_type_of

diff --git a/compiler/rustc_borrowck/src/diagnostics/conflict_errors.rs b/compiler/.../conflict_errors.rs
@@ -2592,4 +2592,4 @@ fn annotate_argument_and_return_for_borrow(
             } else {
-                £\textcolor{black}{let ty = }£self.infcx.tcx.type_of(self.mir_def_id())£\textcolor{black}{;}£
+                £\textcolor{black}{let ty = }£self.infcx.tcx.bound_type_of(self.mir_def_id()).subst_identity()£\textcolor{black}{;}£
                 match ty.kind() {
                     ty::FnDef(_, _) | ty::FnPtr(_) => self.annotate_fn_sig(
diff --git a/compiler/rustc_borrowck/src/diagnostics/mod.rs b/compiler/.../mod.rs
@@ -1185,4 +1185,4 @@ fn explain_captures(
                         matches!(tcx.def_kind(parent_did), rustc_hir::def::DefKind::Impl { .. })
                             .then_some(parent_did)
-                            £\textcolor{black}{.and\_then(|did| match}£ tcx.type_of(did)£\textcolor{black}{.kind() \ttlb}£
+                            £\textcolor{black}{.and\_then(|did| match}£ tcx.bound_type_of(did).subst_identity()£\textcolor{black}{.kind() \ttlb}£
                                 ty::Adt(def, ..) => Some(def.did()),
...
\end{lstlisting}

\textcolor{subtitlex}{136 files changed, 385 insertions(+), 262 deletions(-)}
}

\frame[containsverbatim]{\frametitle{Creating a semantic patch: Step 1: remove irrelevant code}

\begin{lstlisting}[language=tdiff]
££
££
££
££
££
££
££
££
££
-                £\textcolor{black!2}{let ty = }£self.infcx.tcx.type_of(self.mir_def_id())£\textcolor{black!2}{;}£
+                £\textcolor{black!2}{let ty = }£self.infcx.tcx.bound_type_of(self.mir_def_id()).subst_identity()£\textcolor{black!2}{;}£
££
££
££
££
££
££
-                            £\textcolor{black!2}{.and\_then(|did| match}£ tcx.type_of(did)£\textcolor{black!2}{.kind() \ttlb}£
+                            £\textcolor{black!2}{.and\_then(|did| match}£ tcx.bound_type_of(did).subst_identity()£\textcolor{black!2}{.kind() \ttlb}£
££
££
\end{lstlisting}

\textcolor{black!2}{136 files changed, 385 insertions(+), 262 deletions(-)}
}

\frame[containsverbatim]{\frametitle{Creating a semantic patch: Step 2: abstract over subterms using metavariables}

\begin{lstlisting}[language=tdiff]
@@

@@

- self.infcx.tcx.type_of(self.mir_def_id())
+ self.infcx.tcx.bound_type_of(self.mir_def_id()).subst_identity()
\end{lstlisting}

\vfill

\textcolor{black!2}{Updates over 200 call sites.}
}

\frame[containsverbatim]{\frametitle{Creating a semantic patch: Step 2: abstract over subterms using metavariables}

\begin{lstlisting}[language=tdiff]
@@
expression tcx, arg;
@@

- tcx.type_of(arg)
+ tcx.bound_type_of(arg).subst_identity()
\end{lstlisting}

\vfill

\textcolor{black!2}{Updates over 200 call sites.}
}

\frame[containsverbatim]{\frametitle{Creating a semantic patch: Step 2: abstract over subterms using metavariables}

\begin{lstlisting}[language=tdiff]
@@
expression tcx, arg;
@@

- tcx.type_of(arg)
+ tcx.bound_type_of(arg).subst_identity()
\end{lstlisting}

\vfill

\textcolor{subtitlex}{Updates over 200 call sites.}
}

\frame[containsverbatim]{\frametitle{An outlier}

\begin{lstlisting}[language=tdiff]
             let (shim_size, shim_align, _kind) = ecx.get_alloc_info(alloc_id);
+            let def_ty = ecx.tcx.bound_type_of(def_id).subst_identity();
             let extern_decl_layout =
-                ecx.tcx.layout_of(ty::ParamEnv::empty().and(ecx.tcx.type_of(def_id))).unwrap();
+                ecx.tcx.layout_of(ty::ParamEnv::empty().and(def_ty)).unwrap();
             if extern_decl_layout.size != shim_size || extern_decl_layout.align.abi != shim_align {
                 throw_unsup_format!(
                     "`extern` static `{name}` from crate `{krate}` has been declared \
\end{lstlisting}

\textcolor{black!2}{The developer has created a new name to avoid a long
  line.}
\begin{itemize}
\item[] \textcolor{black!2}{Could address it manually.}
\item[] \textcolor{black!2}{Could create a rule for the special case of nested function call
  contexts \newline (probably not worth it for one case).}
\end{itemize}
}

\frame[containsverbatim]{\frametitle{An outlier}

\begin{lstlisting}[language=tdiff]
             let (shim_size, shim_align, _kind) = ecx.get_alloc_info(alloc_id);
+            let def_ty = ecx.tcx.bound_type_of(def_id).subst_identity();
             let extern_decl_layout =
-                ecx.tcx.layout_of(ty::ParamEnv::empty().and(ecx.tcx.type_of(def_id))).unwrap();
+                ecx.tcx.layout_of(ty::ParamEnv::empty().and(def_ty)).unwrap();
             if extern_decl_layout.size != shim_size || extern_decl_layout.align.abi != shim_align {
                 throw_unsup_format!(
                     "`extern` static `{name}` from crate `{krate}` has been declared \
\end{lstlisting}

\textcolor{subtitlex}{The developer has created a new name to avoid a long
  line.}
\begin{itemize}
\item Could address it manually.
\item Could create a rule for the special case of nested function call
  contexts \newline (probably not worth it for one case).
\end{itemize}
}

\frame[containsverbatim]{\frametitle{Alternate semantic patch}

\begin{lstlisting}[language=tdiff]
@@
expression tcx, arg;
@@

  tcx.
-    type_of(arg)
+    bound_type_of(arg).subst_identity()
\end{lstlisting}

\vfill
\textcolor{subtitlex}{Putting tcx in the context ensures any comments will
  be preserved.}
}

\frame[containsverbatim]{\frametitle{An example: change in context}

\begin{lstlisting}[language=tdiff]
commit 1ce80e210d152619caa99b1bc030f57a352b657a
Author: Oliver Scherer <oli-obk@users.noreply.github.com>
Date:   Thu Feb 16 09:25:11 2023 +0000

    Allow `LocalDefId` as the argument to `def_path_str`

diff --git a/compiler/rustc_borrowck/src/lib.rs b/compiler/rustc_borrowck/src/lib.rs
@@ -124,3 +124,3 @@ pub fn provide(providers: &mut Providers) {
 fn mir_borrowck(tcx: TyCtxt<'_>, def: LocalDefId) -> &BorrowCheckResult<'_> {
     let (input_body, promoted) = tcx.mir_promoted(def);
-    debug!("run query mir_borrowck: {}", tcx.def_path_str(def.to_def_id()));
+    debug!("run query mir_borrowck: {}", tcx.def_path_str(def));
diff --git a/compiler/rustc_hir_analysis/src/check/check.rs b/compiler/rustc_hir_analysis/src/check/check.rs
@@ -494,5 +494,5 @@ fn check_item_type(tcx: TyCtxt<'_>, id: hir::ItemId) {
     debug!(
         "check_item_type(it.def_id={:?}, it.name={})",
         id.owner_id,
-        tcx.def_path_str(id.owner_id.to_def_id())
+        tcx.def_path_str(id.owner_id)
     );
...
\end{lstlisting}

\textcolor{subtitlex}{18 files changed, 68 insertions(+), 54 deletions(-)}
}

\frame[containsverbatim]{\frametitle{An example: change in context}

\textcolor{subtitlex}{Want to drop {\tt .to\_def\_id()} but only in an
  argument to {\tt tcx.def\_path\_str}:}

\vspace{0.5\baselineskip}

\begin{lstlisting}[language=tdiff]
@@
expression tcx, arg;
@@

-        tcx.def_path_str(arg.to_def_id())
+        tcx.def_path_str(arg)
\end{lstlisting}

\vfill

Updates 48 call sites in 18 files.}

\frame[containsverbatim]{\frametitle{An example: multiple cases}
\begin{lstlisting}[language=tdiff]
commit 298ae8c721102c36243335653e57a7f94e08f94a
Author: Michael Goulet <michael@errs.io>
Date:   Wed Feb 22 22:23:10 2023 +0000

    Rename ty_error_with_guaranteed to ty_error, ty_error to ty_error_misc

diff --git a/compiler/rustc_borrowck/src/region_infer/opaque_types.rs b/compiler/.../opaque_types.rs
@@ -156,3 +156,3 @@ pub(crate) fn infer_opaque_types(
                     });
-                    prev.ty = infcx.tcx.ty_error_with_guaranteed(guar);
+                    prev.ty = infcx.tcx.ty_error(guar);
                 }
@@ -248,3 +248,3 @@ fn infer_opaque_definition_from_instantiation(
         if let Some(e) = self.tainted_by_errors() {
-            return self.tcx.ty_error_with_guaranteed(e);
+            return self.tcx.ty_error(e);
         }
...
diff --git a/compiler/rustc_hir_analysis/src/astconv/mod.rs b/compiler/rustc_hir_analysis/src/astconv/mod.rs
@@ -429,2 +429,2 @@ fn provided_kind(
                         self.inferred_params.push(ty.span);
-                        tcx.ty_error().into()
+                        tcx.ty_error_misc().into()
\end{lstlisting}

\textcolor{subtitlex}{32 files changed, 121 insertions(+), 140 deletions(-)}
}

\frame[containsverbatim]{\frametitle{An example: multiple cases}
\textcolor{subtitlex}{Two changes:}
\begin{itemize}
\item From {\tt ty\_error\_with\_guaranteed} to {\tt ty\_error} (1 argument)
\item From {\tt ty\_error} to {\tt ty\_error\_misc} (no arguments)
\end{itemize}

\begin{lstlisting}[language=tdiff]
@@
expression tcx, arg;
@@
- tcx.ty_error_with_guaranteed(arg)
+ tcx.ty_error(arg)

@@
expression tcx, arg;
@@
- tcx.ty_error()
+ tcx.ty_error_misc()
\end{lstlisting}
}

\frame[containsverbatim]{\frametitle{An example: searching for variants}
\begin{lstlisting}[language=tdiff]
commit f3f9d6dfd92dfaeb14df891ad27b2531809dd734
Author: Eduard-Mihai Burtescu <edy.burt@gmail.com>
Date:   Fri Jun 14 00:48:52 2019 +0300

    Unify all uses of 'gcx and 'tcx.

diff --git a/src/librustc/infer/error_reporting/mod.rs b/src/librustc/infer/error_reporting/mod.rs
@@ -460,6 +460,6 @@ impl<'gcx, 'tcx> Printer<'gcx, 'tcx> for AbsolutePathPrinter<'gcx, 'tcx> {
             type DynExistential = !;
             type Const = !;
 
-            fn tcx<'a>(&'a self) -> TyCtxt<'gcx, 'tcx> {
+            fn tcx<'a>(&'a self) -> TyCtxt<'tcx> {
                 self.tcx
             }
@@ -1977,4 +1976,4 @@ pub fn enter_global<'gcx, F, R>(gcx: &'gcx GlobalCtxt<'gcx>, f: F) -> R
     pub unsafe fn with_global<F, R>(f: F) -> R
     where
-        F: for<'gcx, 'tcx> FnOnce(TyCtxt<'gcx, 'tcx>) -> R,
+        F: for<'tcx> FnOnce(TyCtxt<'tcx>) -> R,
     {
\end{lstlisting}

\textcolor{subtitlex}{341 files changed, 3109 insertions(+), 3327 deletions(-)}
}

\frame[containsverbatim]{\frametitle{An example: searching for variants}

\textcolor{subtitlex}{A first attempt:}

\begin{lstlisting}[language=tdiff]
@rule type@
@@
- TyCtxt<'gcx, 'tcx>
+ TyCtxt<'tcx>
\end{lstlisting}

\vfill

\textcolor{black!2}{This does part of the work, but some change sites are overlooked:}
\begin{itemize}
\item[] \textcolor{black!2}{{\tt TyCtxt<'tcx, 'tcx>}, {\tt TyCtxt<'\_, '\_>}}
\item[] \textcolor{black!2}{{\tt DepNodeParams<'gcx, 'tcx>}}
\item[] \textcolor{black!2}{And others?}
\end{itemize}}

\frame[containsverbatim]{\frametitle{An example: searching for variants}

\textcolor{subtitlex}{A first attempt:}

\begin{lstlisting}[language=tdiff]
@rule type@
@@
- TyCtxt<'gcx, 'tcx>
+ TyCtxt<'tcx>
\end{lstlisting}

\vfill

\textcolor{subtitlex}{This does part of the work, but some change sites are overlooked:}
\begin{itemize}
\item {\tt TyCtxt<'tcx, 'tcx>}, {\tt TyCtxt<'\_, '\_>}
\item {\tt DepNodeParams<'gcx, 'tcx>}
\item[] \textcolor{black!2}{And others?}
\end{itemize}}

\frame[containsverbatim]{\frametitle{An example: searching for variants}

\textcolor{subtitlex}{A first attempt:}

\begin{lstlisting}[language=tdiff]
@rule type@
@@
- TyCtxt<'gcx, 'tcx>
+ TyCtxt<'tcx>
\end{lstlisting}

\vfill

\textcolor{subtitlex}{This does part of the work, but some change sites are overlooked:}
\begin{itemize}
\item {\tt TyCtxt<'tcx, 'tcx>}, {\tt TyCtxt<'\_, '\_>}
\item {\tt DepNodeParams<'gcx, 'tcx>}
\item And others?
\end{itemize}}

\frame[containsverbatim]{\frametitle{An example: searching for variants}

\textcolor{subtitlex}{A more general attempt:}

\begin{lstlisting}[language=tdiff]
@rule type@
identifier Ty;
@@
- Ty<'gcx, 'tcx>
+ Ty<'tcx>
\end{lstlisting}

\vfill

\textcolor{black!2}{How to find other change sites, like {\tt TyCtxt<'tcx, 'tcx>}, {\tt TyCtxt<'\_, '\_>}:}
\begin{itemize}
\item[] \textcolor{black!2}{Want to change all uses of types that are somewhere used with {\tt 'gcx}.}
\end{itemize}}

\frame[containsverbatim]{\frametitle{An example: searching for variants}

\textcolor{subtitlex}{A more general attempt:}

\begin{lstlisting}[language=tdiff]
@rule type@
identifier Ty;
@@
- Ty<'gcx, 'tcx>
+ Ty<'tcx>
\end{lstlisting}

\vfill

\textcolor{subtitlex}{How to find other change sites, like {\tt TyCtxt<'tcx, 'tcx>}, {\tt TyCtxt<'\_, '\_>}:}
\begin{itemize}
\item Want to change all uses of types that are somewhere used with {\tt 'gcx}.
\end{itemize}}

\frame[containsverbatim]{\frametitle{An example: searching for variants}

\textcolor{subtitlex}{A more general attempt:}

\begin{lstlisting}[language=tdiff]
@r type@
identifier Ty;
@@
- Ty<'gcx, 'tcx>
+ Ty<'tcx>

@rule type@
identifier r.Ty;
@@
(
- Ty<'tcx, 'tcx>
+ Ty<'tcx>
|
- Ty<'_, '_>
+ Ty<'_>
)
\end{lstlisting}

%% \vfill

%% \textcolor{subtitlex}{Takes care of 1287 insertions(+), 1433 deletions(-) \\out of  3109 insertions(+), 3327 deletions(-)}
}

\frame{\frametitle{Summary: Features seen so far}

\begin{itemize}
\setlength{\itemsep}{3mm}
\item Semantic patches: \\Patch-like transformation specification, abstracted using metavariables.
\item Multiple rules/rule ordering.
\item Inheritance.
\item Disjuctions.
\end{itemize}

\vfill

\centerline{\textcolor{subtitlex}{All of these features are implemented!}}}

\frame{\frametitle{Some more Coccinelle features}
\begin{itemize}
\setlength{\itemsep}{2mm}
\item {\tt ...}
\begin{itemize}
\item[--] {\tt A ... B}: {\tt A} can reach {\tt B} through the function control-flow.
\item[--] {\tt <... A ...>}, {\tt <+... A ...+>}: {\tt A} occurs 0+ or 1+ times.
\end{itemize} \pause
\item Position variables.
\item Script code. \pause
\item Constraints on metavariables (types, etc.).
\item Fresh identifiers.
\item {\tt *} for matching without transformation.
\end{itemize}
\vfill

\vspace{0.5\baselineskip}

\centerline{\textcolor{subtitlex}{These features are coming soon...}}}

\frame{\frametitle{Some Coccinelle internals}

\textcolor{subtitlex}{Input:} Parsing provided by Rust Analyzer.
\begin{itemize}
\setlength{\itemsep}{2mm}
\item Used both for Rust code and for semantic patch code.
\item Will provide type inference, when needed.
\end{itemize} \pause

\vspace{\baselineskip}

\textcolor{subtitlex}{Output:} Pretty printing provided by {\tt rustfmt}.
\begin{itemize}
\setlength{\itemsep}{2mm}
\item Preserves comments.
\end{itemize}}

\frame{\frametitle{Some Coccinelle internals}
\textcolor{subtitlex}{In the middle:}
\begin{itemize}
\setlength{\itemsep}{3mm}
\item Wrap Rust code and semantic patch code, eg to indicate metavariables.
\item Match semantic patch code against Rust code, to collect change sites and metavariable bindings.
\item On a successful match, apply the changes, instantiated according to the metavariable bindings, reparse, and repeat with the next rule.
\end{itemize}}

\frame[containsverbatim]{\frametitle{Practical issues}

\begin{lstlisting}[language=diff]
Usage: main [OPTIONS] --coccifile <COCCIFILE> --targetpath <TARGETPATH>

Options:
  -c, --coccifile <COCCIFILE>            Path of Semantic Patch File path
  -t, --targetpath <TARGETPATH>          Path of Rust Target file/folder path
  -o, --output <OUTPUT>                  Path of transformed file path
  -r, --rustfmt-config <RUSTFMT_CONFIG>  rustfmt config file path [default: rustfmt.toml]
  -i, --ignore <IGNORE>                  [default: ]
  -d, --debug-cocci                      
      --apply                            
      --suppress-diff                    
      --suppress-formatting              
      --no-parallel                      
  -h, --help                             Print help
  -V, --version                          Print version
\end{lstlisting}
}

\frame{\frametitle{Conclusion}

\begin{itemize}
\setlength{\itemsep}{3mm}
\item Transformation on atomic terms completed (expressions, types, etc).
\item Transformation on terms connected by a control-flow path ({\tt \ldots}) in progress.
\item Small-scale testing has been done:
\begin{itemize}
\item[--] Replicating real changes on real Rust code.
\end{itemize}
\item Patchparse extended to Rust, to find test cases at a larger scale.
\item Availability: https://gitlab.inria.fr/coccinelle/coccinelleforrust.git
\end{itemize}
}

\end{document}
